
%%%%%%%%%%%%%%%%%%%%%%%%%%%%%%%%%%%%%%%%%%%%%%%%%%%%%%%%%%%%%%%%%%%%%%%%%%%%%%%%%%%%%%%
%%%%%%%%%%%%%%%%%%%%%%%%%%%%%%%%%%%%%%%%%%%%%%%%%%%%%%%%%%%%%%%%%%%%%%%%%%%%%%%%%%%%%%%
% 
% This top part of the document is called the 'preamble'.  Modify it with caution!
%
% The real document starts below where it says 'The main document starts here'.

\documentclass[12pt]{article}

\usepackage{amssymb,amsmath,amsthm}
\usepackage[top=1in, bottom=1in, left=1.25in, right=1.25in]{geometry}
\usepackage{fancyhdr}
\usepackage{enumerate}
\usepackage{listings}

% Comment the following line to use TeX's default font of Computer Modern.
\usepackage{times,txfonts}



\makeatletter
\renewcommand*\env@matrix[1][*\c@MaxMatrixCols c]{%
  \hskip -\arraycolsep
  \let\@ifnextchar\new@ifnextchar
  \array{#1}}
\makeatother

\newtheoremstyle{homework}% name of the style to be used
  {18pt}% measure of space to leave above the theorem. E.g.: 3pt
  {12pt}% measure of space to leave below the theorem. E.g.: 3pt
  {}% name of font to use in the body of the theorem
  {}% measure of space to indent
  {\bfseries}% name of head font
  {:}% punctuation between head and body
  {2ex}% space after theorem head; " " = normal interword space
  {}% Manually specify head
\theoremstyle{homework} 

% Set up an Exercise environment and a Solution label.
\newtheorem*{exercisecore}{Exercise \@currentlabel}
\newenvironment{exercise}[1]
{\def\@currentlabel{#1}\exercisecore}
{\endexercisecore}

\newcommand{\localhead}[1]{\par\smallskip\noindent\textbf{#1}\nobreak\\}%
\newcommand\solution{\localhead{Solution:}}

%%%%%%%%%%%%%%%%%%%%%%%%%%%%%%%%%%%%%%%%%%%%%%%%%%%%%%%%%%%%%%%%%%%%%%%%
%
% Stuff for getting the name/document date/title across the header
\makeatletter
\RequirePackage{fancyhdr}
\pagestyle{fancy}
\fancyfoot[C]{\ifnum \value{page} > 1\relax\thepage\fi}
\fancyhead[L]{\ifx\@doclabel\@empty\else\@doclabel\fi}
\fancyhead[C]{\ifx\@docdate\@empty\else\@docdate\fi}
\fancyhead[R]{\ifx\@docauthor\@empty\else\@docauthor\fi}
\headheight 15pt

\def\doclabel#1{\gdef\@doclabel{#1}}
\doclabel{Use {\tt\textbackslash doclabel\{MY LABEL\}}.}
\def\docdate#1{\gdef\@docdate{#1}}
\docdate{Use {\tt\textbackslash docdate\{MY DATE\}}.}
\def\docauthor#1{\gdef\@docauthor{#1}}
\docauthor{Use {\tt\textbackslash docauthor\{MY NAME\}}.}
\makeatother

% Shortcuts for blackboard bold number sets (reals, integers, etc.)
\newcommand{\Reals}{\ensuremath{\mathbb R}}
\newcommand{\Nats}{\ensuremath{\mathbb N}}
\newcommand{\Ints}{\ensuremath{\mathbb Z}}
\newcommand{\Rats}{\ensuremath{\mathbb Q}}
\newcommand{\Cplx}{\ensuremath{\mathbb C}}
%% Some equivalents that some people may prefer.
\let\RR\Reals
\let\NN\Nats
\let\II\Ints
\let\CC\Cplx

%%%%%%%%%%%%%%%%%%%%%%%%%%%%%%%%%%%%%%%%%%%%%%%%%%%%%%%%%%%%%%%%%%%%%%%%%%%%%%%%%%%%%%%
%%%%%%%%%%%%%%%%%%%%%%%%%%%%%%%%%%%%%%%%%%%%%%%%%%%%%%%%%%%%%%%%%%%%%%%%%%%%%%%%%%%%%%%
% 
% The main document start here.

% The following commands set up the material that appears in the header.
\doclabel{STAT 402: Homework 2}
\docauthor{Stefano Fochesatto}
\docdate{\today}

\begin{document}

\begin{exercise}{1} What is a simple random sample? If all sampling units have the same probability of being
    in the sample, is this always a simple random sample? Why or why not?\\
    \solution A simple random sample(SRS) is a sampling scheme that comes in two varieties, sampling with replacement 
    and sampling without replacement. In a SRS with replacements, every sample is equally likely and after sampling every sampling
    unit is returned to the population with the chance of being sampled again. In a SRS without replacement, after sampling, the sampling 
    unit is not returned to the population. The simple random sample can be performed on a variety of populations where sampling units have 
    different probabilities of being sampled ie. Gaussian Distributed, the key point is that every $sample$ is equally likely. 
\end{exercise}
\vspace{1in}

\begin{exercise}{2} Use R to take a simple random sample of size $n = 4$(without replacement) from this population
    $10,11,13,11,10,6,22,15,14,23$. Please include your output in the homework. Use this sample to compute your estimator of 
    the population mean and find its standard error and a 95 percent confidence interval.

    \solution Note that we are using a SRS without replacement on a relativity small sample, therefore when 
    computing the standard error we must use the finite population correction. \\
    \textbf{Code:}
    \begin{center}
        \lstinputlisting{r1.txt}
    \end{center}
\end{exercise}
\vspace{1in}


\begin{exercise}{3} We want to know the total number of moose in an area. Suppose we have divided 
    the region into $N = 200$ quadrat, our guess is that the standard deviation of moose counts is around $s = 3$
    moose and we would like a margin of error of less than $\pm$ 100 moose. How many sampling units must we visit. \\
    \solution Taking the formula for the margin of error using standard error with the 
    finite population correction and solving it for $n$ we get, 
    \begin{align*}
        n &= \dfrac{N\sigma^2}{(N-1)\frac{B^2}{4N^2} + \sigma^2}\\
          &= \dfrac{200 (3)^2}{(200-1)\frac{100^2}{4 (200)^2} + (3)^2}\\
          &\approx 83.965
    \end{align*}
\end{exercise}
\vspace{1in}


\begin{exercise}{4} Suppose we had decided to sample $n = 12$ of the quadrat and got a sample average of 
    14 moose per quadrat and a sample variance of $s^2 = 125$ square moose per quadrat. Find an estimate of teh total
    number of moose, along with it's standard error and then construct a 95 percent confidence interval for the total number of moose. \\
    \solution Given that there are 100 quadrat, and the expected value for the mean of each quadrat is 14 a good estimator for the total population would
        \begin{equation*}
            t = 100*E(\bar{x}) = 1400.
        \end{equation*}
    Since the SRS at each quadrat are independent we can similarly estimate the variance, 
        \begin{equation*}
            V(t) = 100*V(\bar{x}) = 12500
        \end{equation*}
    Computing the standard error, 
    \begin{equation*}
        SE = \sqrt(12500) = 111.8
    \end{equation*}
    Computing the $95\%$ confidence interval for the total population
    \begin{equation*}
        95\% CI = (1623.6, 1176.4).
    \end{equation*}
\end{exercise}
\vspace{1in}




\end{document}





















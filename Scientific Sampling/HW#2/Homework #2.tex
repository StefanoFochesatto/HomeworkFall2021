
%%%%%%%%%%%%%%%%%%%%%%%%%%%%%%%%%%%%%%%%%%%%%%%%%%%%%%%%%%%%%%%%%%%%%%%%%%%%%%%%%%%%%%%
%%%%%%%%%%%%%%%%%%%%%%%%%%%%%%%%%%%%%%%%%%%%%%%%%%%%%%%%%%%%%%%%%%%%%%%%%%%%%%%%%%%%%%%
% 
% This top part of the document is called the 'preamble'.  Modify it with caution!
%
% The real document starts below where it says 'The main document starts here'.

\documentclass[12pt]{article}

\usepackage{amssymb,amsmath,amsthm}
\usepackage[top=1in, bottom=1in, left=1.25in, right=1.25in]{geometry}
\usepackage{fancyhdr}
\usepackage{enumerate}
\usepackage{listings}

% Comment the following line to use TeX's default font of Computer Modern.
\usepackage{times,txfonts}



\makeatletter
\renewcommand*\env@matrix[1][*\c@MaxMatrixCols c]{%
  \hskip -\arraycolsep
  \let\@ifnextchar\new@ifnextchar
  \array{#1}}
\makeatother

\newtheoremstyle{homework}% name of the style to be used
  {18pt}% measure of space to leave above the theorem. E.g.: 3pt
  {12pt}% measure of space to leave below the theorem. E.g.: 3pt
  {}% name of font to use in the body of the theorem
  {}% measure of space to indent
  {\bfseries}% name of head font
  {:}% punctuation between head and body
  {2ex}% space after theorem head; " " = normal interword space
  {}% Manually specify head
\theoremstyle{homework} 

% Set up an Exercise environment and a Solution label.
\newtheorem*{exercisecore}{Exercise \@currentlabel}
\newenvironment{exercise}[1]
{\def\@currentlabel{#1}\exercisecore}
{\endexercisecore}

\newcommand{\localhead}[1]{\par\smallskip\noindent\textbf{#1}\nobreak\\}%
\newcommand\solution{\localhead{Solution:}}

%%%%%%%%%%%%%%%%%%%%%%%%%%%%%%%%%%%%%%%%%%%%%%%%%%%%%%%%%%%%%%%%%%%%%%%%
%
% Stuff for getting the name/document date/title across the header
\makeatletter
\RequirePackage{fancyhdr}
\pagestyle{fancy}
\fancyfoot[C]{\ifnum \value{page} > 1\relax\thepage\fi}
\fancyhead[L]{\ifx\@doclabel\@empty\else\@doclabel\fi}
\fancyhead[C]{\ifx\@docdate\@empty\else\@docdate\fi}
\fancyhead[R]{\ifx\@docauthor\@empty\else\@docauthor\fi}
\headheight 15pt

\def\doclabel#1{\gdef\@doclabel{#1}}
\doclabel{Use {\tt\textbackslash doclabel\{MY LABEL\}}.}
\def\docdate#1{\gdef\@docdate{#1}}
\docdate{Use {\tt\textbackslash docdate\{MY DATE\}}.}
\def\docauthor#1{\gdef\@docauthor{#1}}
\docauthor{Use {\tt\textbackslash docauthor\{MY NAME\}}.}
\makeatother

% Shortcuts for blackboard bold number sets (reals, integers, etc.)
\newcommand{\Reals}{\ensuremath{\mathbb R}}
\newcommand{\Nats}{\ensuremath{\mathbb N}}
\newcommand{\Ints}{\ensuremath{\mathbb Z}}
\newcommand{\Rats}{\ensuremath{\mathbb Q}}
\newcommand{\Cplx}{\ensuremath{\mathbb C}}
%% Some equivalents that some people may prefer.
\let\RR\Reals
\let\NN\Nats
\let\II\Ints
\let\CC\Cplx

%%%%%%%%%%%%%%%%%%%%%%%%%%%%%%%%%%%%%%%%%%%%%%%%%%%%%%%%%%%%%%%%%%%%%%%%%%%%%%%%%%%%%%%
%%%%%%%%%%%%%%%%%%%%%%%%%%%%%%%%%%%%%%%%%%%%%%%%%%%%%%%%%%%%%%%%%%%%%%%%%%%%%%%%%%%%%%%
% 
% The main document start here.

% The following commands set up the material that appears in the header.
\doclabel{STAT 402: Homework 2}
\docauthor{Stefano Fochesatto}
\docdate{\today}

\begin{document}

\begin{exercise}{1} What is a simple random sample? If all sampling units have the same probability of being
    in the sample, is this always a simple random sample? Why or why not?\\
    \solution A simple random sample(SRS) is a sampling scheme that comes in two varieties, sampling with replacement 
    and sampling without replacement. In a SRS with replacements, every sample is equally likely and after sampling every sampling
    unit is returned to the population with the chance of being sampled again. In a SRS without replacement, after sampling, the sampling 
    unit is not returned to the population. The simple random sample can be performed on a variety of populations where sampling units have 
    different probabilities of being sampled ie. Gaussian Distributed, the key point is that every $sample$ is equally likely. 
\end{exercise}
\vspace{1in}

\begin{exercise}{2} Use R to take a simple random sample of size $n = 4$(without replacement) from this population
    $10,11,13,11,10,6,22,15,14,23$. Please include your output in the homework. Use this sample to compute your estimator of 
    the population mean and find its standard error and a 95 percent confidence interval.

    \solution Note that we are using a SRS without replacement on a relativity small sample, therefore when 
    computing the standard error we must use the finite population correction. \\
    \textbf{Code:}
    \begin{center}
        \lstinputlisting{r1.txt}
    \end{center}
\end{exercise}
\vspace{1in}


\begin{exercise}{3} We want to know the total number of moose in an area. Suppose we have divided 
    the region into $N = 200$ quadrat, our guess is that the standard deviation of moose counts is around $s = 3$
    moose and we would like a margin of error of less than $\pm$ 100 moose. How many sampling units must we visit. \\
    \solution Taking the formula for the margin of error using standard error with the 
    finite population correction and solving it for $n$ we get, 
    \begin{align*}
        n &= \dfrac{N\sigma^2}{(N-1)\frac{B^2}{4N^2} + \sigma^2}\\
          &= \dfrac{200 (3)^2}{(200-1)\frac{100^2}{4 (200)^2} + (3)^2}\\
          &\approx 83.965
    \end{align*}
\end{exercise}
\vspace{1in}


\begin{exercise}{4} Suppose we had decided to sample $n = 12$ of the quadrat and got a sample average of 
    14 moose per quadrat and a sample variance of $s^2 = 125$ square moose per quadrat. Find an estimate of teh total
    number of moose, along with it's standard error and then construct a 95 percent confidence interval for the total number of moose. \\
    \solution Given that there are 100 quadrat, and the expected value for the mean of each quadrat is 14 a good estimator for the total population would
        \begin{equation*}
            t = 100*E(\bar{x}) = 1400.
        \end{equation*}
    Since the SRS at each quadrat are independent we can similarly estimate the variance, 
        \begin{equation*}
            V(t) = 100*V(\bar{x}) = 12500
        \end{equation*}
    Computing the standard error, 
    \begin{equation*}
        SE = \sqrt(12500) = 111.8
    \end{equation*}
    Computing the $95\%$ confidence interval for the total population
    \begin{equation*}
        95_{CI} = (1623.6, 1176.4).
    \end{equation*}
\end{exercise}
\vspace{1in}


\begin{exercise}{5} I wish to estimate the total number of squirrels in a large region. I'll do that by 
    dividing the region into $N = 1500$ transects, each 10m wide and 1km long. I select $n = 10$ of these 
    to visit and count animals. I'll assume that I count everyone in the transect and ignore all animals 
    outside the transect. I get the following counts: 12, 20 ,8, 42, 23, 18, 6, 8, 13, 17.\\
    \begin{enumerate}
        \item Find an estimator of the total number of squirrels in the entire region and the standard error, along with a 95 percent confidence interval.\\
        \solution We proceed with the same procedure as the last problem. First we need an estimator for the mean number of squirrels in one transect. 
        \begin{equation*}
            \bar{x} = 16.7
        \end{equation*}
        Computing the estimate for the total population, 
        \begin{equation*}
            t = 1500\bar{x} = 25050
        \end{equation*}
        Computing it's variance, 
        \begin{equation*}
            V(t) = 1500V(\bar{x}) =  165683.3
        \end{equation*}
        Computing the SE and $95\%$ confidence interval we get, 
        \begin{equation*}
            SE_T = \sqrt{165683.3} = 407.04
        \end{equation*}
        \begin{equation*}
            95_{CI} = (25864.08, 24235.92)
        \end{equation*}
        \vspace{.15in}

        \item If I divide the estimated total by the total area of the region I'll get the density in squirrels. Find 
        a 95 percent confidence interval for this density.\\
        \solution The area of each transect is $1/100$km given the linearly of the expected value our estimate for the population total density is, 
        \begin{equation*}
            t_{density} = \dfrac{1500\bar{x}}{100} = 250.5
        \end{equation*}
        Because of the law of variances we need to square the $1/100$ coefficient when computing the variance of our estimate, 
        \begin{equation*}
            V(t_{density}) = \dfrac{1500V(\bar{x})}{100^2} \approx 16.7.
        \end{equation*}
        Computing the SE and $95\%$ confidence interval,
        \begin{equation*}
            SE_T = \sqrt{16.7} = 4.07,
        \end{equation*}
        \begin{equation*}
            95_{CI} = (258.6408 242.3592).
        \end{equation*}
    \end{enumerate}
\end{exercise}
\vspace{1in}


\begin{exercise}{6} To use the typical estimator $\pm 2$ standard errors to find a 95 percent interval 
    for a proportion, you need the sampling distribution of the sample proportion to be close to normal. You can assume 
    this is the case if $n*(est proportion)> 10$ and $n*(1 - est proportion)> 10$. Suppose we are looking for the proportion of 
    spruce trees in a low land forest that have a certain genetic trait. We somehow make a list of $N=1320$ trees in the area. 
    We take a SRS of size $n = 120$ trees and find tha 13 have the genetic trait. \\
    \begin{enumerate}
        \item Is this sample size sufficient for us to assume the sampling distribution of $\hat{p}$ is approximately normal?
        why or why not?\\
        \solution First we need to identify our sample proportion, 
        \begin{equation*}
            \hat{p} = \dfrac{13}{120}.
        \end{equation*}
        Based on the test described above we do have a large enough sample size to assume sampling distribution of $\hat{p}$ is approximately normal,
        \begin{equation*}
            120*\dfrac{13}{120} = 13 > 10.
        \end{equation*}
        \vspace{.15in}

        \item Find a $95\%$ confidence interval for the true proportion of trees in the region with the trait.\\
        \solution First we need to compute the standard error, 
        \begin{equation*}
            \hat{p}_{SE} = \sqrt{\dfrac{1320 - 120}{1320}\dfrac{\frac{13}{120}(1 - \frac{13}{120})}{120-1}} \approx 0.0271.
        \end{equation*}
        So we get a $95\%$ confidence interval of,
        \begin{equation*}
            95_{CI} = (.1625, .0541).
        \end{equation*}
    \end{enumerate}    
\end{exercise}
\vspace{.25in}

\begin{exercise}{7} I wish to conduct a political poll in a small town. the town has a total of 
    450 residents ans I can actually take a SRS of them. I would really like a margin of error of, at most, 
    $\pm$ .05. What sample size should I take? What sample size would I need to take if I perversely decided to use 
    SRS WITH replacement.\\
    \solution Assuming the worst case, being that the true proportion is split 50/50 or at least close to it as most political polls are
    we can compute the sample size without replacement with the following formula, 
    \begin{equation*}
        n = \dfrac{N}{(N-1)B^2 + 1} = \dfrac{450}{(450-1)(.05)^2 + 1} \approx 213.
    \end{equation*}
    In the case that we take a SRS with replacement we essentially have an infinite population size, and the formula for 
    finding $n$ reduces to, 
    \begin{equation*}
        n = \frac{1}{B^2}=\frac{1}{.05^2}= 400. 
    \end{equation*}
\end{exercise}
\vspace{.25in}

\begin{exercise}{8} Use bootstrapping on the data in problem 2 to get a 95 percent confidence interval for the population 
    mean.\\
    \solution 
    \textbf{Code:}
    \begin{center}
        \lstinputlisting{r2.txt}
    \end{center}
\end{exercise}
\vspace{.25in}

\begin{exercise}{9} Take a SRS(without replacement) of size $n = 18$ from the population in the table. 
    Find the 95 percent confidence interval for the total over teh entire area. Later you will compare this estimate with the estimate from a stratified random sample.\\
    \solution 
    \textbf{Code:}
    \begin{center}
        \lstinputlisting{r3.txt}
    \end{center}
\end{exercise}
\vspace{.25in}





\end{document}






















%%%%%%%%%%%%%%%%%%%%%%%%%%%%%%%%%%%%%%%%%%%%%%%%%%%%%%%%%%%%%%%%%%%%%%%%%%%%%%%%%%%%%%%
%%%%%%%%%%%%%%%%%%%%%%%%%%%%%%%%%%%%%%%%%%%%%%%%%%%%%%%%%%%%%%%%%%%%%%%%%%%%%%%%%%%%%%%
% 
% This top part of the document is called the 'preamble'.  Modify it with caution!
%
% The real document starts below where it says 'The main document starts here'.

\documentclass[12pt]{article}

\usepackage{amssymb,amsmath,amsthm}
\usepackage[top=1in, bottom=1in, left=1.25in, right=1.25in]{geometry}
\usepackage{fancyhdr}
\usepackage{enumerate}


% Comment the following line to use TeX's default font of Computer Modern.
\usepackage{times,txfonts}

\newtheoremstyle{homework}% name of the style to be used
  {18pt}% measure of space to leave above the theorem. E.g.: 3pt
  {12pt}% measure of space to leave below the theorem. E.g.: 3pt
  {}% name of font to use in the body of the theorem
  {}% measure of space to indent
  {\bfseries}% name of head font
  {:}% punctuation between head and body
  {2ex}% space after theorem head; " " = normal interword space
  {}% Manually specify head
\theoremstyle{homework} 

% Set up an Exercise environment and a Solution label.
\newtheorem*{exercisecore}{Exercise \@currentlabel}
\newenvironment{exercise}[1]
{\def\@currentlabel{#1}\exercisecore}
{\endexercisecore}

\newcommand{\localhead}[1]{\par\smallskip\noindent\textbf{#1}\nobreak\\}%
\newcommand\solution{\localhead{Solution:}}

%%%%%%%%%%%%%%%%%%%%%%%%%%%%%%%%%%%%%%%%%%%%%%%%%%%%%%%%%%%%%%%%%%%%%%%%
%
% Stuff for getting the name/document date/title across the header
\makeatletter
\RequirePackage{fancyhdr}
\pagestyle{fancy}
\fancyfoot[C]{\ifnum \value{page} > 1\relax\thepage\fi}
\fancyhead[L]{\ifx\@doclabel\@empty\else\@doclabel\fi}
\fancyhead[C]{\ifx\@docdate\@empty\else\@docdate\fi}
\fancyhead[R]{\ifx\@docauthor\@empty\else\@docauthor\fi}
\headheight 15pt

\def\doclabel#1{\gdef\@doclabel{#1}}
\doclabel{Use {\tt\textbackslash doclabel\{MY LABEL\}}.}
\def\docdate#1{\gdef\@docdate{#1}}
\docdate{Use {\tt\textbackslash docdate\{MY DATE\}}.}
\def\docauthor#1{\gdef\@docauthor{#1}}
\docauthor{Use {\tt\textbackslash docauthor\{MY NAME\}}.}
\makeatother

% Shortcuts for blackboard bold number sets (reals, integers, etc.)
\newcommand{\Reals}{\ensuremath{\mathbb R}}
\newcommand{\IRats}{\ensuremath{\mathbb I}}
\newcommand{\Nats}{\ensuremath{\mathbb N}}
\newcommand{\Ints}{\ensuremath{\mathbb Z}}
\newcommand{\Rats}{\ensuremath{\mathbb Q}}
\newcommand{\Cplx}{\ensuremath{\mathbb C}}
%% Some equivalents that some people may prefer.
\let\RR\Reals
\let\NN\Nats
\let\II\Ints
\let\CC\Cplx


\doclabel{STAT 401}
\docauthor{Stefano Fochesatto}
\docdate{\today}% DATE READING SENTENCES ARE DUE GOES HERE


%%%% Main document starts here.

\begin{document}

\section*{Introductory Assignment}

\begin{description}
\item[\#1:] What is your name and what are you studying?\\ \\
My name is Stefano Fochesatto and I am studying math.  

\newpage
%applied geometry and the theories regarding the great
\item[\#2:] Why are you taking STAT 401 and what skills or knowledge would you like to gain from this class?\\ \\
  I recognize that in today's world data is incredibly valuable for solving and recognizing problems we face as a society. Being able to create, interpret and 
  fit models to data is an instrumental skill in being able to solve those problems. In general I hope this class can add another facet to my problem solving skills.

  \newpage

\item[\#3:]What experience do you have with R?\\ \\
  I have used R in the past. We used it multiple times in Math 371 Probability to complete labs. I am however much more familiar with Matlab and Python. From my experience 
  programming skills are fairly transferable between these languages so I would maybe classify myself as an intermediate R user. 
  \newpage

\item[\#4:]Is there anything else you would like me to know?\\ \\
  I'm very excited to start school!

  \newpage

\item[\#5:]Refer to the “self-transcendent purpose” article posted in Canvas.  
What is one (or more) specific self-transcendent purpose(s) you have for your schooling?  Answers will be only skimmed during
grading and graded merely for completeness—the point of this exercise is get you thinking about this topic for your own benefit.\\ \\
  I was really drawn towards Natalie Semmel's story. When she describes her success at New Haven Academy with "I was really able to thrive knowing that's where our values as a school were."
  it really outlined the difference between good leadership, which inspires self-transcendent purpose and bad leadership, which fails to do so. For myself, I think that one of the biggest problems we face as a 
  society today is wealth inequality, which in turn leads to an ineffective and destructive distribution of social capitol. Brilliant people move to lucrative industries that are bad for the environment and our society, all
  in an effort to make a living wage.  



\end{description}
\end{document}

%%%%%%%%%%%%%%%%%%%%%%%%%%%%%%%%%%%%%%%%%%%%%%%%%%%%%%%%%%%%%%%%%%%%%%%%%%%%%%%%%%%%%%%
%%%%%%%%%%%%%%%%%%%%%%%%%%%%%%%%%%%%%%%%%%%%%%%%%%%%%%%%%%%%%%%%%%%%%%%%%%%%%%%%%%%%%%%
% 
% This top part of the document is called the 'preamble'.  Modify it with caution!
%
% The real document starts below where it says 'The main document starts here'.

\documentclass[12pt]{article}

\usepackage{amssymb,amsmath,amsthm}
\usepackage[top=1in, bottom=1in, left=1.25in, right=1.25in]{geometry}
\usepackage{fancyhdr}
\usepackage{enumerate}
\usepackage{listings}

% Comment the following line to use TeX's default font of Computer Modern.
\usepackage{times,txfonts}



\makeatletter
\renewcommand*\env@matrix[1][*\c@MaxMatrixCols c]{%
  \hskip -\arraycolsep
  \let\@ifnextchar\new@ifnextchar
  \array{#1}}
\makeatother

\newtheoremstyle{homework}% name of the style to be used
  {18pt}% measure of space to leave above the theorem. E.g.: 3pt
  {12pt}% measure of space to leave below the theorem. E.g.: 3pt
  {}% name of font to use in the body of the theorem
  {}% measure of space to indent
  {\bfseries}% name of head font
  {:}% punctuation between head and body
  {2ex}% space after theorem head; " " = normal interword space
  {}% Manually specify head
\theoremstyle{homework} 

% Set up an Exercise environment and a Solution label.
\newtheorem*{exercisecore}{Exercise \@currentlabel}
\newenvironment{exercise}[1]
{\def\@currentlabel{#1}\exercisecore}
{\endexercisecore}

\newcommand{\localhead}[1]{\par\smallskip\noindent\textbf{#1}\nobreak\\}%
\newcommand\solution{\localhead{Solution:}}

%%%%%%%%%%%%%%%%%%%%%%%%%%%%%%%%%%%%%%%%%%%%%%%%%%%%%%%%%%%%%%%%%%%%%%%%
%
% Stuff for getting the name/document date/title across the header
\makeatletter
\RequirePackage{fancyhdr}
\pagestyle{fancy}
\fancyfoot[C]{\ifnum \value{page} > 1\relax\thepage\fi}
\fancyhead[L]{\ifx\@doclabel\@empty\else\@doclabel\fi}
\fancyhead[C]{\ifx\@docdate\@empty\else\@docdate\fi}
\fancyhead[R]{\ifx\@docauthor\@empty\else\@docauthor\fi}
\headheight 15pt

\def\doclabel#1{\gdef\@doclabel{#1}}
\doclabel{Use {\tt\textbackslash doclabel\{MY LABEL\}}.}
\def\docdate#1{\gdef\@docdate{#1}}
\docdate{Use {\tt\textbackslash docdate\{MY DATE\}}.}
\def\docauthor#1{\gdef\@docauthor{#1}}
\docauthor{Use {\tt\textbackslash docauthor\{MY NAME\}}.}
\makeatother

% Shortcuts for blackboard bold number sets (reals, integers, etc.)
\newcommand{\Reals}{\ensuremath{\mathbb R}}
\newcommand{\Nats}{\ensuremath{\mathbb N}}
\newcommand{\Ints}{\ensuremath{\mathbb Z}}
\newcommand{\Rats}{\ensuremath{\mathbb Q}}
\newcommand{\Cplx}{\ensuremath{\mathbb C}}
%% Some equivalents that some people may prefer.
\let\RR\Reals
\let\NN\Nats
\let\II\Ints
\let\CC\Cplx

%%%%%%%%%%%%%%%%%%%%%%%%%%%%%%%%%%%%%%%%%%%%%%%%%%%%%%%%%%%%%%%%%%%%%%%%%%%%%%%%%%%%%%%
%%%%%%%%%%%%%%%%%%%%%%%%%%%%%%%%%%%%%%%%%%%%%%%%%%%%%%%%%%%%%%%%%%%%%%%%%%%%%%%%%%%%%%%
% 
% The main document start here.

% The following commands set up the material that appears in the header.
\doclabel{STAT 401: Homework 1}
\docauthor{Stefano Fochesatto}
\docdate{\today}

\begin{document}



\begin{exercise}{1} Below are three sample of a random variable. Which sample displays the 
    greatest variance? Why?
    \begin{align*}
        &\text{Sample 1: 21.45, 22.93, 31.86, 19.37, 20.87}\\
        &\text{Sample 2: 182, 186, 179, 184, 187}\\
        &\text{Sample 3: 14.3, 14.7, 10.0, 14.8, 14.6}
    \end{align*}

    \solution I would say, just from looking at the samples number 3 has the lowest variance since 4 of the 
    entries seemed to be grouped around 14.5 and the only outlier is 10, which isn't very far either. Out of samples
    1 and 2 I'm not really sure which one has greater variance but I would guess that its sample 1 since the 31.86 term 
    is a very large (relatively) outlier, while the entries in sample 2 are all more or less equally spaced. 
\end{exercise}
\newpage

\begin{exercise}{2} A random sample of dogs studied by veterinarians was found to have 
    the following lifespans in years: ${14,18, 6,9,13}$. The veterinarians inquired about each
    dog's mother and found the mother' lifespans to be: ${10, 19, 13,8,24}$. What is the sample covariance
    between the offsprings' lifespans and mothers' lifespans?\\
    \solution 
    First we need to compute the sample mean for each of our samples. 
    \begin{equation*}
        \bar{x_{dogs}} = \dfrac{14+18+6+9+13}{5} = 12,
    \end{equation*}
    \begin{equation*}
        \bar{x_{mothers}} = \dfrac{10+19+13+8+24}{5} = 14.8,
    \end{equation*}
    Now we can compute the sample covariance,
    \begin{equation*}
        Cov(X_{dogs},X_{mothers}) = \dfrac{\sum(x_{dogs}-\bar{x_{dogs})}(x_{mothers}-\bar{x_{mothers})}}{n-1} = 14
    \end{equation*}
I omitted the large fraction and algebra, but I double checked it with r and it does check out. 
\end{exercise}
\newpage

\begin{exercise}{3} Consider $X$, a random variable with a standard normal distribution. 
    What is the probability that $X$ is greater than $-.4$.\\
    \solution So given that $X \approx N(1,0)$ we want to find the value of $P(X>-.4)$. Using r or a table we 
    can compute the probability, and we get that, 
    \begin{equation*}
        P(X>-.4)\approx 0.655
    \end{equation*}
    \textbf{Code:}
		\begin{center}
			\lstinputlisting{r1.txt}
		\end{center}
\end{exercise}

\newpage

\begin{exercise}{4} Consider $V$, a random variable with a $N(13,400)$. What is the probability 
    that $V$ is less than or equal to $-7$?\\
    \solution First we want to get our probability in terms of the standard normal, 
    \begin{equation*}
        P(V\leq -7) = P\left(\frac{V - 13}{\sqrt{400}} \leq \frac{-7 - 13}{\sqrt{400}}\right) = P(Z \leq -1)
    \end{equation*}
    Using r or a table we can compute the probability, and we get that, 
    \begin{equation*}
        P(V\leq -7) \approx 0.159
    \end{equation*}
    \textbf{Code:}
		\begin{center}
			\lstinputlisting{r2.txt}
		\end{center}
\end{exercise}
\newpage


\begin{exercise}{5} Let the expected value of random variable $X$ be $a$, the expected value of 
    $Y$ be $b$, and the expected value of $Z$ be c. Find $E(4-2X+3Y-10Z)$.\\
    \solution Applying the rule of the expected values we get, 
    \begin{equation*}
        E(4-2X+3Y-10Z) = 4-2E(X)+3E(Y)-10E(Z).
    \end{equation*} 
    By substitution we get, 
    \begin{equation*}
        E(4-2X+3Y-10Z) = 4-2a+3b-10c.
    \end{equation*}
\end{exercise}
\newpage


\begin{exercise}{6} Let the variance of random variable $X$ be 3, the variance of 
    $Y$ be 12, and the variance of $Z$ be 9, and let $X$, $Y$, and $Z$ be uncorrelated
    . Find $V(4-2X+3Y-10Z)$.\\
    \solution Applying the rule of variances we get,
    \begin{equation*}
        V(4-2X+3Y-10Z) = V(4)+(-2)^2V(X)+3^2V(Y)+(-10)^2V(Z)
    \end{equation*}
    By substitution we get, 
    \begin{equation*}
        V(4-2X+3Y-10Z) = V(4)+2^2(3)+3^2(12)+10^2(9) = V(4)+1020.
    \end{equation*}
    Note that, $V(4) = E(4^2) - E(4)^2 = 16 - 16 = 0$ so we get that,
    \begin{equation*}
        V(4-2X+3Y-10Z) = 1020.
    \end{equation*}

\end{exercise}
\newpage




\begin{exercise}{7} If a random variable $A$ is distributed $N(3,8)$ and an independent random variable 
    $B$ is distributed $N(-3,10)$, what is the distribution of $mA+nB$ for some constants $m$ and $n$?\\
    \solution Recall that if $A$ and $B$ are independent random variables that are normally distributed, their 
    sum is also normally distributed. Therefore we know that $mA+nB$ is a normally distributed random variable.
    Solving for $E(mA+nB)$,
    \begin{equation*}
        E(mA+nB) = E(mA)+E(nB) = E(A)m+E(B)n = 3m-3n = 3(m-n)
    \end{equation*}
    Solving for $V(mA+nB)$, 
    \begin{equation*}
        V(mA+nB) = V(mA)+ V(nB) = V(A)m^2 + V(B)n^2 = 8m^2 + 10n^2
    \end{equation*}
    So finally we get that $mA+nB$ is distributed $N(3(m-n),8m^2 + 10n^2)$.
\end{exercise}
\newpage



\begin{exercise}{8} Angie records the lengths of 15 poems from ancient Greece and finds the average wordcount
    of the poems is 456 with a standard deviation of 192. What is the standard error of Angie's estimator of 
    the mean wordcount for all Greek poems?\\
    \solution Recall the formula for computing the standard error of the sample mean, 
    \begin{equation*}
        SE_{\overline{x}} = \frac{S}{\sqrt{n}}
    \end{equation*}
    By substitution we get that, 
    \begin{equation*}
        SE_{\overline{x}} = \frac{192}{\sqrt{15}} \approx 48.57 
    \end{equation*}
\end{exercise}
\newpage



\begin{exercise}{9} Consider the mean of a random sample of size 75, $\overline{X}$. If $S^2$ is the 
    sample variance and the population is normally distributed with mean $\mu$, what is the distribution of,
    \begin{equation*}
        \dfrac{\overline{X} - \mu}{S/\sqrt{75}}
    \end{equation*}
    \solution The formula above describes the normalizing of a random sample. If the estimators 
    were replaced with their population parameters we would have a variable with a standard normal distribution, instead
    we have a variable with a t-distribution with 74 degrees of freedom. 
\end{exercise}
\newpage



\begin{exercise}{10} The mean weight of peanuts in a sample of size 16 from a barrel is 0.09 ounces. 
    The standard deviation of the sample is 0.015 ounces. What is a $90\%$ confidence interval for the mean weight of 
    all peanuts in the barrel? Assume peanut weight s in the barrel are normally distributed. 
    \solution 
Recall we compute the $90\%$ CI with the following formula, 
\begin{equation*}
    90\%CI_{\overline{x}} = \overline{x} \pm (t_{.1/2, n-1} \cdot \frac{s}{\sqrt{n}})
\end{equation*}
Finding the critical  t-value for a $90\%CI$ with 15 degrees of freedom using r or a table we get,  
\begin{equation*}
    t_{.5, 15} = 1.75305 
\end{equation*}
Now we can find the confidence interval, 
\begin{align*}
    90\%CI_{\overline{x}} &= .09 \pm ( 1.75305 \cdot \frac{.015}{\sqrt{16}})\\
    &\approx .09 \pm 0.00657
\end{align*}
\end{exercise}
\newpage


\begin{exercise}{11} The weights, in pounds, of a team of sixteen male athletes are as follows:
    \begin{equation*}
        188.5, 183.0, 194.5, 185.0, 214.0, 203.5, 186.0, 178.5 ,186.0, 184.5, 204.0, 184.5, 195.5, 202.5, 174.0, 183.0
    \end{equation*}
    Assume that this team is representative of all athletes in the league. Test the hypothesis that the 
    mean weight of the league is greater than 190. Assume that athlete weights in the league are normally
    distributed and use an $\alpha$ level of 0.01.\\
    \solution 
    First we must define our experiment, 
    \begin{align*}
        H_0&: \mu = 190\\
        H_a&: \mu > 190
    \end{align*}
    Computing the sample mean and standard deviation we get that, 
    \begin{align*}
        \overline{x} &= 190.4375,\\
        s&=10.84416
    \end{align*}
    Now we need to solve for the one-sample $t$ test statistic,
    \begin{equation*}
        t = \dfrac{\overline{x} - 190}{s/\sqrt(n)} = 0.1613772
    \end{equation*}
    To compute the p-value we need to find the area to the right of $t$ in the 
    $T_{n-1}$ distribution. Using r or a table we get, 
    \begin{equation*}
        p \approx 0.4369749
    \end{equation*}
    Since $p>.01$ we fail to reject the null hypothesis. There is not sufficient evidence 
    at the $1\%$ level to conclude that the mean athlete weight is greater than 190 pounds. 

\textbf{Code:}
    \begin{center}
        \lstinputlisting{r3.txt}
    \end{center}
\end{exercise}
\newpage


\begin{exercise}{12} A doctor is performing a clinical trian of research medication
    for treating symptoms of emphysema. She recruits 50 patients with the disease and randomly 
    assigns 25 to the treatment group and 25 to the placebo group. After 6 weeks, she measures 
    each patient's blood oxygen level and compares the sample means of the two groups. She calculates 
    a test statistic of 4.38 and a p-value of 0.0001. She concludes that, since $0.0001<0.025$, the 
    medication significantly raises blood oxygen levels compared to the placebo. She calculates that the 
    power of her test is .7.\\

    What would constitute a type I error in this scenario and what is it;s probability? What would 
    constitute a type II error in this scenario and what is it's probability?

    \solution Let's consider the doctors experiment, suppose that $\mu_1$ is the average blood oxygen 
    level of the medication users and $\mu_2$ is the placebo group. 
    \begin{align*}
        H_0&: \mu_1 = \mu_2\\
        H_a&: \mu_1 > \mu_2
    \end{align*}
    A type I error constitutes the doctor concluding that the medication does alleviate symptoms
    of emphysema, when in reality it does not. The probability of this happening is $\alpha = .025$\\

    A type II error constitutes the doctor concluding that the medication fails to alleviate symptoms of 
    emphysema, when in reality it does. The probability of this happening is $\beta = 1-power = .3$

\end{exercise}


\end{document}




















